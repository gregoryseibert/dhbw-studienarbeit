%!TEX root = ../hauptdatei.tex
\chapter{Implementierung}
In diesem Kapitel wird beschrieben, wie die Backend Komponente und die zwei Frontend Komponenten, eine iOS App sowie ein Web Admin-Dashboard, als Referenz Implementierung entwickelt wird. 

\section{Backend} \label{backend}
Das Backend des Bäckerei Systems ist die zentrale Datenverwaltung, durch die neue Datensätze angelegt oder bestehende Datensätze abgerufen, bearbeitet oder gelöscht werden können. Um neue Datensätze nachhaltig speichern zu können, besitzt das Backend eine Datenbank.
\\
Im Folgenden wird beschrieben, aus welchen Hauptbestandteilen sich das Spring Backend zusammensetzt.
Diese Hauptbestandteile sind die Repository Klassen, die Service Klassen und die Controller Klassen.
Die Controller Klassen nutzen die Service Klassen und die Service Klassen nutzen die Repository Klassen als Abstraktionsschicht über der Datenbank.

\clearpage

\subsubsection{Repository Ebene}
Wie bereits beschrieben, stellt die Repository Ebene eine direkte Abstraktionsschicht der Datenbank dar.
Um eine Repository Klasse erstellen zu können, muss einem Java Interface die Annotation \code{@Transactional} zugewiesen werden. Das Interface muss außerdem das Interface \code {CrudRepository} erweitern und den zu behandelnden Objekttyp sowie den Datentyp des Attributs zur Identifikation angeben. Dies ist im folgenden Codeblock als Beispiel aufgeführt.
\begin{lstlisting}[language=Java]
@Transactional
public interface ExampleRepository extends CrudRepository<Example, Long> {
	//optional repository methods
}
\end{lstlisting}

Hierbei sind Methoden zum Speichern, Aktualisieren und Löschen. von Objekten bereits standardmäßig definiert.
Da nach Kapitel \ref{datenmodell} die Objekte der Klassen \code{Loaf}, \code{Bun}, \code{Ingredient}, \code{CerealMixPercentage}, \code{NewsItem} und \code{ApplicationUser} verwaltet werden müssen, werden entsprechende Repository Klassen entwickelt.
Um später im Objektbestand der Klassen \code{Loaf}, \code{Bun} und \code{Ingredient} nach dem Namen suchen zu können, ist eine Methode \enquote{findByName} (siehe folgenden Codeblock) notwendig.
\begin{lstlisting}[language=Java]
Example findByName(String name);
\end{lstlisting}

Des Weiteren soll es möglich sein. einzelne Objekte oder alle Objekte der Klassen \code{Loaf}, \code{Bun}, \code{Ingredient}, \code{CerealMixPercentage}, \code{NewsItem} und \code{ApplicationUser} abzurufen.
Hierfür sind die Methoden \code{findById} und \code{findAll} (siehe folgenden Codeblock) notwendig.
\begin{lstlisting}[language=Java]
Example findById(long id);
List<Example> findAll();
\end{lstlisting}

Für die Klasse \enquote{ApplicationUser} soll es zudem möglich sein, ein Objekt über den Benutzernamen mit der Methode \code{findByUsername} zu finden sowie mit der Methode \enquote{existsByUsername} zu überprüfen, ob bereits ein Benutzer mit einem spezifischen Benutzernamen existiert (siehe folgenden Codeblock).
\begin{lstlisting}[language=Java]
User findByUsername(String username);
boolean existsByUsername(String username);
\end{lstlisting}
Die nach diesem Muster implementierten Repository Klassen mit den jeweiligen Methoden für \code{Loaf}, \code{Bun}, \code{Ingredient}, \code{CerealMixPercentage}, \code{NewsItem} und \code{ApplicationUser} sind in der Abbildung \ref{backend-repositories} als UML Klassendiagramm ersichtlich.

\abbildung{../diagramme/Backend-Repositories.pdf}{Die Repository Ebene des Backends}{backend-repositories}

\clearpage

\subsubsection{Service Ebene}
Die Service Ebene verwendet die Repository Ebene als Abstraktionsschicht über der Datenbank, um vordefinierte Methoden zum Erstellen, Anzeigen, Bearbeiten, und Löschen von Datensätzen zu ermöglichen. Diese Service Klassen werden anschließend von Controller Klassen verwendet und stellen somit selbst eine Abstraktionsschicht dar.
Um eine Klasse als einen Spring Service deklarieren zu können, ist die Verwendung der Annotation \code{@Service} nowendig.
\begin{lstlisting}[language=Java]
@Service
public class ExampleService {
	//service methods
}
\end{lstlisting}

Da eine Referenz zur jeweiligen Instanz einer Repository Klasse notwendig ist, wird diese durch die Annotation \code{@Autowired} per Dependency Injection bereitgestellt. Analog gilt des auch für die ModelMapper Klasse, um das Architekturmuster \gls{dto} zu unterstützen.
\begin{lstlisting}[language=Java]
@Autowired
private ModelMapper modelMapper;

@Autowired
private ExampleRepository exampleRepository;
\end{lstlisting}

Hiermit ist die Implementierung neuer Funktionen in den Service Klassen mit Minimalaufwand möglich, wie in dem folgenden Listing zu sehen ist.
\begin{lstlisting}[language=Java]
public Example updateExample(long id, ExampleDTO exampleDTO) {
        Example existingExample = getExampleById(id);

        modelMapper.map(exampleDTO, existingExample);

        return exampleRepository.save(existingExample);
    }
\end{lstlisting}

Nach diesem Schema werden die Service Klassen für \code{Loaf}, \code{Bun}, \code{Ingredient}, \code{CerealMixPercentage}, \code{NewsItem} und \code{ApplicationUser} implementiert. Des Weiteren ist eine Service Klasse mit dem Namen \code{FileSystemStorageService} notwendig, welche die Speicherung von Daten auf dem Dateisystem ermöglicht (siehe Abbildung \ref{backend-services}). 

\clearpage

\abbildung{../diagramme/Backend-Services.pdf}{Die Service Ebene des Backends}{backend-services}

\clearpage

\subsubsection{Controller Ebene}
Mit der Repository Ebene ist bereits eine Abstraktionsschicht über der Datenbank und mit der Service Ebene ist eine Abstraktionsschicht für bereitgestellte Funktionalitäten vorhanden. Um nun mit deren Hilfe eine \gls{rest} Schnittstelle anbieten zu können, ist die Controller Ebene notwendig.
Eine Controller Klasse kann mittels der Annotation \code{@RestController} erstellt werden, wobei der Methoden übergreifende Basis-Endpunkt durch die Annotation \code{@RequestMapping} deklariert werden kann.
\begin{lstlisting}[language=Java]
@RestController
@RequestMapping("/example")
public class ExampleController {
	//controller methods
}
\end{lstlisting}

Eine Methode kann nun mit Minimalaufwand nach dem folgenden Beispiel implementiert werden.
\begin{lstlisting}[language=Java]
@GetMapping
public List<Example> getAllExamples() {
        return exampleService.getAllExamples();
}
\end{lstlisting}

Damit eine Methode lediglich durch Administratoren aufgerufen werden können, ist die Annotation \code{PreAuthorize} notwendig.
\begin{lstlisting}[language=Java]
@PostMapping
@PreAuthorize("hasRole('ROLE_ADMIN')")
public Example createExample(@RequestBody ExampleDTO exampleDTO) {
	return exampleService.createExample(exampleDTO);
}
\end{lstlisting}

Nach diesem Schema werden die Controller Klassen für \code{Loaf}, \code{Bun}, \code{Ingredient}, \code{CerealMixPercentage}, \code{NewsItem} und \code{ApplicationUser} implementiert (siehe Abbildung \ref{backend-controllers}). 

\clearpage

\abbildung{../diagramme/Backend-Controllers.pdf}{Die Controller Ebene des Backends}{backend-controllers}

\clearpage

\section{iOS Applikation}
Damit die Kunden der Bäckerei die im System gespeicherten Daten zu den Produkten anschaulich durchsuchen können, wird eine iOS Applikation entwickelt.
Die Referenz-Implementierung hierzu wird in diesem Kapitel erklärt.

\subsection{Architektur}
Die Architektur der Applikation ist nach dem \gls{mvc} Muster, mit einer Modell Ebene, einer View Ebene und einer Controller Ebene, aufgebaut. Dieses Muster wurde bereits im Kapitel \ref{mvc} erläutert.
Zur Entkopplung der Abhängigkeiten wird die Logik zur Kommunikation mit dem Backend und zur lokalen Speicherung der Daten als eigene Schicht, die Datasource Ebene, abstrahiert.
Dies bietet den Vorteil, im späteren Verlauf die Datenquelle auszutauschen, ohne die Klassen der Controller Ebene im großen Maße abzuändern. 

\subsubsection{Datasource Ebene}
Zur lokalen Speicherung der Daten ist die Klasse \code{Disk} zuständig.
Sie besitzt die private Methode \code{getURL}, um die \gls{url} zurückzugeben, welche durch iOS für das angegebene Verzeichnis, \code{documents} oder \code{caches}, zur Verfügung gestellt wird und die den spezifischen Speicherort repräsentiert.
Ihre private Methode \code{storeData} speichert die übergebenen Daten unter dem angegebenen Verzeichnis, indem sie die Funktion \code{getURL} verwendet, ab.
Die Funktion \code{storeCodable}  sorgt für die Umwandlung eines Objekts, dessen Klassen dem \code{Codable} Protokoll entspricht, in Daten und speichert diese Daten durch Verwendung der privaten Funktion \code{storeData} und unter der Angabe des Verzeichnisses ab.
Damit bereits gespeicherte Daten abgerufen werden können, besitzt sie die private Funktion \code{fetchData}. Diese wird von der Funktion \code{fetchCodable} verwendet, um aus den ausgelesenen Daten wiederum Objekte der angegebenen Klasse zu generieren.
Des Weiteren sind Methoden zum Löschen aller Inhalte eines Verzeichnisses mittels \code{clear}, zum Löschen einer Datei mittels \code{remove} und zum Überprüfen der Existenz einer Datei mittels \code{fileExist} implementiert.

\clearpage

Die Klasse \code{RestApiHandler}, ist für die direkte Kommunikation mit einem \gls{rest} Backend zuständig und nutzt hierfür die Swift internen Bibliotheken.
Über die Methode \code{init}, welche in Swift den Konstruktor repräsentiert, wird die Basis \gls{url} gesetzt, welche für alle Anfragen verwendet wird.
Durch ihre private Funktion \code{createRequest} (nicht authentifiziert) oder \code{createAuthenticatedRequest} (authentifiziert) können Anfragen erstellt werden.Damit die authentifizierten Anfragen jedoch durch das Backend auch als solche erkannt werden, muss vorher der Anmeldeprozess mittels der Funktion \code{login} durchlaufen sein.
Die beiden Methoden \code{getDecodableArrayFromEndpoint}, zum Abfragen von Objekten, und \code{getDataFromEndpoint}, zum Abfragen von Ressourcen wie zum Beispiel Bilder, können zum Ausführen von Anfragen verwendet werden.
\\
Die Klasse \code{BakeryDatasource} verwendet die Klasse \code{RestApiHandler}, um spezialisierte Methoden bereitzustellen.
Hierzu gehört \code{login} zum Anmelden sowie \code{existingBakedGoodsOnDisk} zum Überprüfen, ob auf dem lokalen Speicher Objekte der Klasse \code{BakedGood} vorhanden sind.
Des Weiteren werden die Funktionen \code{getBakedGoodsFromAPI} und \code{getBakedGoodsFromDisk}, zum Abrufen von Objekten der Klasse \code{BakedGood} vom Backend oder dem lokalen Speicher, bereitgestellt. Analog dazu werden die Funktionen \code{getNewsItemsFromAPI} und \code{getNewsItemsFromDisk} für die Klasse \code{NewsItem} bereitgestellt.

\subsubsection{Modell Ebene}
Die Modellklassen werden analog zum Datenmodell im Kapitel \ref{datenmodell} implementiert.
Diese Klassen implementieren das Protokoll  \code{Codable}, damit Swift die \gls{json} Daten in Objekte automatisch konvertieren kann. Eine beispielhafte Klasse ist im folgenden Codeblock ersichtlich.
\begin{lstlisting}[language=Swift]
class BakedGood: Codable {
    let id: Int
    var name, characteristic, pictureFilename: String
    var weight, kcal, fat, carbohydrates, protein: Double
    var cerealMix: [CerealMixPercentage]
    var ingredients: [Ingredient]
}
\end{lstlisting}

\clearpage

\subsubsection{View Ebene}
Die Klassen dieser View Ebene werden zum Großteil durch Swift bereitgestellt, so dass nur minimale Anpassungen durch spezialisierende Klassen notwendig sind.
Es wird eine Klasse \code{ImageLabelCollectionViewCell} benötigt, welche eine Spezialisierung der Klasse \code{UICollectionViewCell} darstellt und somit die Zellen einer \code{CollectionView} repräsentiert und sie designt. Instanzen einer \code{CollectionView} besitzen eine Zeile-Spalten Struktur jener Zellen.
Des Weiteren ist die Klasse \code{ColumnFlowLayout} , die die Klasse \code{UICollectionViewFlowLayout} spezialisiert, notwendig, um die Größe der Zellen zu berechnen und zu setzen.

\subsubsection{Controller Ebene}
Die Controller Ebene ist, wie bereits im Kapitel \ref{mvc} erläutert, die Zwischenstelle zur Modell und View Ebene.
Für jede Anzeige von Daten, wie zum Beispiel Brote oder Neuigkeiten, ist ein eigener ViewController notwendig.
Dies lässt sich nochmals in die Controller zur Anzeige von allen Datensätze der jeweiligen Kategorie (Masteransicht) und in die Controller zur Anzeige eines einzelnen Datensatzes (Detailansicht) aufteilen.

\clearpage

\subsection{Anzeige von Broten}
Um zur Masteransicht für die Brote (siehe Abbildung \ref{app-bun} auf der linken Seite) zu gelangen, kann man in jeder Ansicht der App das Element mit der Beschriftung \enquote{Loafs} in der unteren Navigationsleiste auswählen.
In dieser Ansicht werden alle im System hinterlegten Brote mit der jeweiligen Bezeichnung und dem Bild angezeigt.
Hier kann sich der Benutzer der App einen groben Überblick verschaffen und nach Belieben ein Brot auswählen, indem er auf die jeweilige Zelle tippt.
Damit der Benutzer bei einer großen Anzahl an Broten schnell das gewünschte Exemplar finden kann, befindet sich oben unterhalb des Logos eine Suchleiste mit einer Filterfunktion (siehe Kapitel \ref{app-filter}).
Wurde ein Brot ausgewählt, so wird die Detailansicht (siehe Abbildung \ref{app-bun} auf der rechten Seite) mit der Bezeichnung, dem Bild, dem Getreidemischungsverhältnis, den Zutaten, dem Gewicht, den Produktionstagen, den Nährwerten und der Charakteristik angezeigt.
\abbildung{images/content/ios-app/app-loaf.png}{Die Masteransicht und Detailansicht zu den Broten}{app-loaf}

\clearpage

\subsection{Anzeige von Brötchen}
Wählt der Benutzer das Element der Navigationsleiste mit der Beschriftung \enquote{Buns} aus, so wird die Masteransicht für die Brötchen (siehe Abbildung \ref{app-loaf} auf der linken Seite) geöffnet. Hier kann sich der Benutzer alle verfügbaren Brötchen anschauen und in diesem Bestand suchen und filtern (siehe Kapitel \ref{app-filter}). Auch in dieser Masteransicht wird beim Auswählen eines Exemplars die Detailansicht (siehe Abbildung \ref{app-loaf} auf der rechten Seite) geöffnet.
In dieser Detailansicht werden die jeweiligen Informationen der Bezeichnung, dem Getreidemischungsverhältnis, den Zutaten, dem Gewicht, den Nährwerten und der Charakteristik angezeigt.

\abbildung{images/content/ios-app/app-bun.png}{Die Masteransicht und Detailansicht zu den Brötchen}{app-bun}

\clearpage

\subsection{Suchen und filtern von Broten oder Brötchen} \label{app-filter}
Bei einem großen Bestand an verschiedenen Broten und Brötchen wird die Übersichtlichkeit beim Anschauen der jeweiligen Masteransicht verringert. Damit der Benutzer weiterhin schnell das gewünschte Exemplar anschauen kann, befindet sich eine Suchleiste im oberen Bereich der App. Wird diese ausgewählt, öffnet sich die Tastatur und es werden fünf Tabs mit Allergenen angezeigt, über die der Nutzer Produkte nach seiner Allergie filtern kann.

\abbildungs{12cm}{images/content/ios-app/app-search.png}{Die Masteransicht und Detailansicht zu den Brötchen}{app-search}

\clearpage

\subsection{Anzeige von Neuigkeiten}
Wird das Element der Navigationsleiste mit der Beschriftung \enquote{News} ausgewählt, so wird die Masteransicht für die Neuigkeiten (siehe Abbildung \ref{app-news} auf der linken Seite) geöffnet. Hier kann sich der Benutzer alle Neuigkeiten in absteigender Reihenfolge anschauen und bei Bedarf sich die detaillierte Ansicht  (siehe Abbildung \ref{app-news} auf der rechten Seite) mit der zusätzlichen Beschreibung öffnen lassen.

\abbildung{images/content/ios-app/app-news.png}{Die Masteransicht und Detailansicht zu den Neuigkeiten}{app-news}

\clearpage

\section{Web Admin-Dashboard}
Das Web Admin-Dashboard bietet dem Administrator die Möglichkeit, die vorhandenen Daten zu den Backprodukten zu pflegen und neue Daten anzulegen. Dieses Dashboard wird als React Web Applikation entwickelt und nutzt die \gls{rest} Schnittstelle des Backends, welches im Kapitel \ref{backend} beschrieben wurde. Um mit dem Backend kommunizieren zu können, wird die Bibliothek \fnurl{Axios}{https://github.com/axios/axios} verwendet.

\subsection{Architektur}
Die Bestandteile des Dashboards werden aufgeteilt, indem mehrfach verwendbare Komponenten extrahiert werden und als eigene Klassen in die Component Ebene einziehen. Diese Komponenten werden anschließend von den sogenannten \enquote{Pages} verwendet, welche die jeweilige Seite, wie zum Beispiel das Erstellen von Broten, darstellen. 

\subsubsection{Component Ebene} \label{dashboard-components}
Zum Erstellen sowie Bearbeiten von Backprodukten wird die Komponente \code{BakedGoodCreateEditCard} entwickelt. Bei deren Verwendung wird eine Liste an verfügbaren Zutaten sowie die Funktion, welche beim Absenden des Formulars aufgerufen werden soll, übergeben. Optional lässt sich angeben, ob das Formular die Produktionstage beinhalten soll und ob es sich um das Bearbeiten eines bestehenden Backprodukts handelt.
\\
Damit Backprodukte in der Masteransicht dargestellt werden können, wird für die einzelnen Elemente die Komponente \code{BakedGoodMasterCard} implementiert. Analog hierzu wird \code{BakedGoodDetailCard} für die Detailansicht eines einzelnen Elements verwendet.
\\
Um Zutaten für die Backprodukte anlegen zu können, wird analog zu den Backprodukten die Komponente \code{IngredientCreateEditCard} erstellt.
\\
Des Weiteren wird eine Komponente für die Navigationsleiste sowie eine zum Benachrichtigen über den Erfolg oder das Fehlschlagen von Aktivitäten, wie zum Beispiel bei der Eingabe von falschen Anmeldedaten, implementiert.

\clearpage

\subsubsection{Page Ebene}
Unter Verwendung der Komponenten aus dem vorigen Kapitel \ref{dashboard-components} wird für jede Seite zur Administration der Daten eine \enquote{Page} implementiert. Hierzu gehören die Seiten
\begin{itemize}
	\item \code{Home} Die Startseite des Dashboards
	\item \code{Login} Die Seite zum Anmelden mittels Formular
	\item \code{Logout} Die Seite zum Abmelden
	\item \code{BunCreate} Stellt ein Formular zum Erstellen von Brötchen bereit
	\item \code{BunDetail} Präsentiert die detaillierten Informationen zu einem Brötchen
	\item \code{BunEdit} Bietet die Möglichkeit an, die Daten eines Brötchens zu bearbeiten
	\item \code{BunMaster} Präsentiert alle vorhandenen Brötchen
	\item \code{IngredientCreate} Über diese Seite lassen sich neue Zutaten erstellen
	\item \code{IngredientEdit} Zuständig für das Bearbeiten von bestehenden Daten einer Zutat
	\item \code{IngredientMaster} Hier werden alle vorhandenen Zutaten aufgelistet
	\item \code{LoafCreate} Per Formular lässt sich auf dieser Seite ein Brot erstellen
	\item \code{LoafDetail} Präsentiert die detaillierten Informationen zu einem Brot
	\item \code{LoafEdit} Bietet die Möglichkeit an, die Daten eines Brotes zu bearbeiten
	\item \code{LoafMaster} Liste alle vorhandenen Brote auf
\end{itemize}

\clearpage

\subsection{Anmelden}
Bevor die Seiten zum Administrieren der Inhalte des Systems aufgerufen werden können, muss sich ein Benutzer des Dashboards erst über die Anmeldeseite, siehe Abbildung \ref{dashboard-login}, authentifizieren. Hierfür muss der Benutzername und das Passwort eingegeben und klassisch auf den Login-Button geklickt werden. War die Anmeldung erfolgreich, so bekommt der Benutzer eine Erfolgsmeldung und die Seite wird neugeladen, so dass die geschützten Seiten aufrufbar werden. Bei einer erfolglosen Anmeldung bekommt der Benutzer eine Fehlermeldung und kann einen erneuten Versuch starten. Wird ein nicht authentifizierter Zugriff auf eine der geschützten Seite versucht, so erfolgt eine automatische Umleitung auf die Anmeldeseite.
\abbildung{images/content/admin-dashboard/dashboard-login.png}{Die Anmeldeseite mit einem Formular für den Benutzernamen und das zugehörige Passwort}{dashboard-login}

\clearpage

\subsection{Erstellen und bearbeiten von Brötchen}
Um neue Brötchen anlegen oder vorhandene bearbeiten zu können, muss in der Navigationsleiste der Link mit der Beschriftung \enquote{Buns} ausgewählt werden. Es wird nun eine Liste mit allen vorhandenen Brötchen dargestellt. Wenn ein neues Brötchen angelegt werden soll, so kann man den dafür vorgesehenen Button benutzen. Ansonsten wählt man ein vorhandenes Brötchen aus und klickt auf \enquote{Edit}. Sobald die Seite mit dem Formular geladen ist, siehe Abbildung \ref{dashboard-create-edit-bun} kann man die gewünschten Informationen eingeben oder ändern sowie das zugehörige Bild hochladen. Zum Schluss klickt man auf den Button mit der Beschriftung \enquote{CREATE} oder \enquote{UPDATE}, je nachdem ob es sich um ein neues oder bestehendes Brötchen handelt.

\abbildung{images/content/admin-dashboard/dashboard-create-edit-bun.png}{Die Seite, um neue Brötchen erstellen und bearbeiten zu können}{dashboard-create-edit-bun}

\clearpage

\subsection{Erstellen und bearbeiten von Broten}
Zum Erstellen oder Bearbeiten von Broten klickt man auf den Navigationslink namens \enquote{Loafs}. Alle bereits angelegten Brote werden in einer Liste dargestellt. Hier kann man einzelne Brote auswählen, um sie zu bearbeiten oder ein neues Brot anlegen. Analog zum Erstellen oder Bearbeiten von Brötchen, lassen sich auch hier die gewünschten Informationen in das Formular, siehe Abbildung \ref{dashboard-create-edit-loaf}, eingeben oder bei Belieben abändern. Der Vorgang wird beim Klicken auf den untersten Button abgeschlossen.

\abbildung{images/content/admin-dashboard/dashboard-create-edit-loaf.png}{Die Seite, um neue Brote erstellen und bearbeiten zu können}{dashboard-create-edit-loaf}

\clearpage

\subsection{Erstellen und bearbeiten von Zutaten}
Bevor neue Zutaten erstellt oder bestehende bearbeitet werden können, muss der Navigationslink mit der Beschriftung \enquote{Ingredients} ausgewählt werden. Anschließend hat man bei der Auflistung aller bestehenden Zutaten die Wahl, ob eine spezifische Zutat bearbeitet werden oder ob eine neue angelegt werden soll. Ist die Wahl erfolgt, lassen sich die Informationen in das Formular nach Abbildung \ref{dashboard-create-edit-ingredient} eingeben und durch das Betätigen des farblich hervorgehobenen Button abspeichern.

\abbildung{images/content/admin-dashboard/dashboard-create-edit-ingredient.png}{Die Seite, um neue Zutaten erstellen und bearbeiten zu können}{dashboard-create-edit-ingredient}
