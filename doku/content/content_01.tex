%!TEX root = ../hauptdatei.tex
\chapter{Einleitung}
Die aktuelle Gesetzeslage in Deutschland fordert, dass eine Information zu den jeweiligen Nährwerten und Allergenen von Lebensmitteln mündlich, schriftlich oder elektronisch zu erfolgen ist \cite{bmel}. Bei einer mündlichen Information müssen jedoch die Daten weiterhin in schriftlicher Form vorliegen \cite{bmel}. Da diese Produkthefte aus bedrucktem Papier bestehen, ist das Pflegen von neuen oder veränderten Daten äußerst mühsam und aufwendig.
\\
Ziel dieser Arbeit ist, ein System zu entwickeln, welches für Endkunden einer Bäckerei eine iOS Applikation bereitstellt, so dass sie sich über Backprodukte und deren Allergene und Nährwerte sowie über Neuigkeiten zur Bäckerei informieren können. Hierbei soll eine Suche nach Produkten sowie eine Filterung nach Allergenen zur Verfügung stehen. Diese Daten werden zentral in einem Backend gespeichert und können mittels einer Schnittstelle angelegt, abgerufen, bearbeitet oder gelöscht werden.
\\
Im zweiten Kapitel werden die theoretische Grundlagen erklärt, welche für das Verständnis der folgenden Kapitel notwendig sind.
Die funktionalen sowie nichtfunktionalen Anforderungen werden im dritten Kapitel getroffen, um als Referenz für das Konzept zu dienen. Das Konzept zu dem zu entwickelnden System wird im Kapitel vier erarbeitet und diskutiert sowie im Kapitel fünf als Referenz Implementierung dargestellt. Im sechsten Kapitel wird das Ergebnis dieser Arbeit zusammengefasst sowie reflektiert.